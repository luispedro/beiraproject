\documentclass[article,twoside]{memoir}
\usepackage[latin9]{inputenc}
\usepackage{psfrag}
\usepackage{hyperref}
\usepackage{eepic}
\usepackage{color}
\usepackage{amsmath}
\usepackage{amssymb}
\usepackage{bm}
\usepackage{amsfonts}
\usepackage{amsthm}
\usepackage{sublabel}
\usepackage[vlined,linesnumbered,ruled]{algorithm2e}
\usepackage[]{graphicx}

\graphicspath{{../figures/other/}{../figures/generated/}}


\DeclareMathOperator*{\argmax}{arg\,max}
\DeclareMathOperator*{\argmin}{arg\,min}

\newcommand{\pd}[2]{\frac{\partial#1}{\partial#2}}
\newcommand{\pdd}[2]{\frac{\partial^2 #1}{\partial #2^2}} 
\newcommand{\pdpd}[3]{\frac{\partial^2 #1}{\partial #2 \partial #3}} 
\newcommand{\B}{\ensuremath{\mathbb{B}}}
\newcommand*{\Pa}{\ensuremath{\text{\upshape\textbf{Pa}}}}
\newcommand*{\Dkl}{\ensuremath{D_{\text{\textsc{kl}}}}}
\newcommand*{\encspace}{\quad}

\newtheorem{theorem}{Theorem}
\newtheorem{lemma}{Lemma}
\newtheorem{definition}{Definition}

\newcommand*{\slantfrac}[2]{\hbox{$\raisebox{-.4ex}{$\,^#1$}\!/_#2$}}
\newcommand*{\onehalf}{\ensuremath{
{\frac{1}{2}}%
}}
\newcommand*{\dx}{\,dx}
\newcommand*{\dt}{\,dt}

\newcommand*{\Expected}{\ensuremath{\mathbb{E}}}
\newcommand*{\Var}{\ensuremath{\text{Var}}}
\newcommand*{\Reals}{\ensuremath{\mathbb{R}}}
\newcommand*{\Binary}{\ensuremath{\mathbb{B}}}
\newcommand*{\discr}{\mbox{discr}}
\newcommand*{\from}{\leftarrow}

\newcommand*{\indicator}[1]{\hspace{1pt}[\hspace{-.4em}[\hspace{3pt} #1 \hspace{3pt}]\hspace{-.4em}]\hspace{2pt}}
\newcommand*{\Assign}{\ensuremath\,:=\,}
\newcommand*{\vect}[1]{\ensuremath{\bm{#1}}}
\newcommand*{\textvalue}[1]{\mbox{\textsl{#1}}}
\newcommand*{\bigO}{\mathcal{O}}
\newcommand*{\MI}{\mbox{MI}}

\DeclareMathOperator{\goodness}{score}
\DeclareMathOperator{\powerset}{Pow}

\title{Theatre Classes in Mozambique}
\author{Rita Reis \and Luis Pedro Coelho}

\bibliographystyle{alpha}
%\bibliographystyle{plain}
\pagestyle{Ruled}

\aliaspagestyle{chapter}{Ruled}
\makeatletter
\if@twoside
	\makeoddhead{Ruled}{}{}{\scshape\rightmark}
	\makeevenhead{Ruled}{\scshape\leftmark}{}{}
	\makeevenfoot{Ruled}{\pagenumberfont\thepage}{}{}
	\makeoddfoot{Ruled}{}{}{\pagenumberfont\thepage}
	% Put section number on top
	\def\sectionmark#1{\markright{#1 (\thesection)}}
	\def\chaptermark#1{\markboth{#1}{#1}}
	\renewcommand*{\bibmark}{\markboth{\bibname}{\bibname}} % I don't like empty headings!
\else

	% Put section number on top
	\def\sectionmark#1{\markright{#1 (\thesection)}}
	\def\chaptermark#1{\markright{#1 (\thechapter)}}

	\makeoddhead{Ruled}{}{}{\scshape\rightmark}
	\makeevenhead{Ruled}{}{}{\scshape\rightmark}
	\makeevenfoot{Ruled}{}{}{\pagenumberfont\thepage}
	\makeoddfoot{Ruled}{}{}{\pagenumberfont\thepage}
	\renewcommand*{\bibmark}{\markright{\bibname}} % I don't like empty headings!


\fi
\makeatother

%Change fonts: Page number sans-serif (much cleaner than the roman version)
\def\pagenumberfont{\sffamily}
% Change fonts: Section headers Sans-Serif:
\setsecheadstyle{\Large\sffamily\raggedright}
\setsubsecheadstyle{\large\sffamily\raggedright}
\setsubsubsecheadstyle{\normalsize\sffamily\raggedright}

% Title
\pretitle{\LARGE\sffamily}
\posttitle{\par\vspace{4ex}}
\preauthor{\large\sffamily\hspace{1cm}}
\postauthor{\par\vspace{3ex}}
\predate{\small\sffamily\hspace{1cm}Last Updated on: }
\postdate{\par\vspace{2cm}}
\copypagestyle{title}{plain}
\makeoddfoot{title}{}{}{}
\makeevenfoot{title}{}{}{}

\renewcommand{\abstractnamefont}{\sffamily}
\renewcommand{\abstracttextfont}{}
\renewcommand{\absnamepos}{flushleft}


\makechapterstyle{mestrado}{% Originally ``AlexanderGrebenkov'', adapted
\renewcommand{\chapterheadstart}{\goodbreak\vspace*{\beforechapskip}\medskip}
\renewcommand{\chapnamefont}{\normalfont\Large\scshape}
\renewcommand{\chapnumfont}{\normalfont\Large\scshape}
\renewcommand{\chaptitlefont}{\normalfont\Large\scshape}
\renewcommand{\printchaptername}{}
\renewcommand{\chapternamenum}{}
\renewcommand{\printchapternum}{\normalfont\Large\scshape\S\thechapter}
\renewcommand{\afterchapternum}{\hspace{1em}}
\renewcommand{\afterchaptertitle}{\par\nobreak\vspace{-.9em}\moveright 6pt\vbox to 1pt{\hrule width .64\textwidth}\nobreak\vskip\afterchapskip\nobreak}
}
\chapterstyle{mestrado}

\SetCommentSty{textsl}
\begin{document}
\begin{abstract}
In June and July 2010, Rita Reis and Luis Pedro Coelho, traveled to Mozambique
and taught theatre classes to high school students in four rural schools. Each
workshop was a week-long affair, with students working 1h30 per day.

Most of the students involved were already participating in school theatre
groups, which in the Mozambican context is closely linked to working as
volunteers for different organisations to spread information on topics such as
HIV/AIDS, other STDs, or education.

The classes were a mix of warm-up exercises for students (some of which formed
a core, repeated almost every day); text work, and improv work. In some cases,
there was interest in developing English language skills and some of the work
was done in English (some schools had English language theatre groups already,
organised by the Peace Corps volunteers---an American volunteering programme).

The programme described here is the result of an evolution throughout our stay,
as we had the opportunity to work in different mission with two groups each
time. We were able to modify the presentation and exercises themselves as we
perceive them to be more or less successful (there were, of course, other
modifications that were made in response to the specificity of each group). We
describe our experience here in the hope that it will be useful to someone else
in the future.

\end{abstract}

\chapter{Introduction}
\chapter{Exercise Catalogue}
\section{Core}
\begin{itemize}
\item Spinal awakening,
\item spinal twists,
\item boom-cha-boom,
\item Oy, steady, take-it,
\item balance of the space.
\end{itemize}

\subsection{Spinal Awakening}
\subsection{Spinal Twists}
\subsection{Boom Cha Boom}
This was adapted in every school to include the name of the school, followed by
Sofala (the province) or Mozambique. The main text ``We've got the rythm in our
hands|feet|hips|eyes'' was sometimes translated into Portuguese, sometimes not,
depending on the general level of the students.

\begin{verse}
Ma-chan-ga, boom-cha-boom\\
So-fa-la, boom-cha-boom\\
temos-o-rit-mo-nas-mãos/pés/ancas/olhos.

\subsection{Oy, steady, take-it}
This was adapted into Portuguese, as Oi-toma-espelho. An early attempt at using
the English version led to much confusion on part of the students.

\section{Improvs}

In some groups, improvs were in English (including at the request of the
students), sometimes it was in mixed language, sometimes we allowed them to use
an invented gibbership if they wished (but forbade Portuguese and
Ndau\footnote{Ndau is the local language. In truth, if the students chose to
speak Ndau, they would have been able to get away with it as neither of us
understood it.}). The students might not have always stuck to the rules, but
they did practice their English skills and some were much more fluent on stage
than when they were talking to the instructor.

Group improv settings were ``chapa'', health care centre, market, and party.

In Mozambique, a ``chapa'' is a 15-person minivan which serves as public
transportation carrying up to 20 passenger crammed into its seats. It is used
both in the city (where it is actually a very convenient way to get around as
the frequency is very high) and inter-city. The ticket collector, who also
calls out for passengers whenever the chapa stops (chapas stop on any
intersection as long as someone inside or outside calls out, but there are
semi-established stops), served as the leader in this improv. The stage started
with him alone on stage and he would call out to the rest of the group to join
him. When we had assembled a sufficient number of passengers, he'd call out for
a driver (this mimicks the behaviour of actual chapas at terminal points when
the driver might step out for a little break as the ticket-collector gathers
new passengers). Other characters include salespersons, traffic police (who,
this being Mozambique, were willing for a bribe), the beggar, chapa owner, and,
in one inspired scene, the man ran over by the chapa. We describe it here at
length as it was invariably successful in terms of both including everyone,
being fun, and generating interesting scenes for the instructors watching.

\section{Other Exercises}
\subsection{Object Transformation}
An object (such as a pencil-case) was passed around and pupils were asked to
make it as if it was something else (e.g., a mirror, a shoe,\dots). In some of
the classes, the remaining students would call out loud the name of the object
in English. This thus served as a vocabulary exercise.

\chapter{Workshop Structure}
\section{Day I}
\section{Day II}
\section{Day III}
\section{Day IV}
\section{Day V}

\chapter{Challenges}
\section{Language}

Mozambique is home to many languages. In the region of Sofala where we were
working, Ndau is the most common language (Sena is also spoken in Sofala).
Portuguese is the official language of the country, but, particularly in the
country-side, some students only encounter it when they reach school age.
Therefore, their levels vary.


\chapter{Assets}
As well as challenges, Mozambicans had many assets when compared to Western
students of the same age.

\chapter{Conclusions}


\end{document}
