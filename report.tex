\documentclass[article,twoside]{memoir}
\usepackage[latin9]{inputenc}
\usepackage{psfrag}
\usepackage{hyperref}
\usepackage{color}
\usepackage[]{graphicx}

\graphicspath{{../figures/other/}{../figures/generated/}}


\title{Theatre for Development: Teaching Theatre in Rural Mozambique}
\author{Rita Reis and Luis Pedro Coelho}

\bibliographystyle{alpha}
%\bibliographystyle{plain}
\pagestyle{Ruled}

\aliaspagestyle{chapter}{Ruled}
\makeatletter
\if@twoside
	\makeoddhead{Ruled}{}{}{\scshape\rightmark}
	\makeevenhead{Ruled}{\scshape\leftmark}{}{}
	\makeevenfoot{Ruled}{\pagenumberfont\thepage}{}{}
	\makeoddfoot{Ruled}{}{}{\pagenumberfont\thepage}
	% Put section number on top
	\def\sectionmark#1{\markright{#1 (\thesection)}}
	\def\chaptermark#1{\markboth{#1}{#1}}
	\renewcommand*{\bibmark}{\markboth{\bibname}{\bibname}} % I don't like empty headings!
\else

	% Put section number on top
	\def\sectionmark#1{\markright{#1 (\thesection)}}
	\def\chaptermark#1{\markright{#1 (\thechapter)}}

	\makeoddhead{Ruled}{}{}{\scshape\rightmark}
	\makeevenhead{Ruled}{}{}{\scshape\rightmark}
	\makeevenfoot{Ruled}{}{}{\pagenumberfont\thepage}
	\makeoddfoot{Ruled}{}{}{\pagenumberfont\thepage}
	\renewcommand*{\bibmark}{\markright{\bibname}} % I don't like empty headings!


\fi
\makeatother

%Change fonts: Page number sans-serif (much cleaner than the roman version)
\def\pagenumberfont{\sffamily}
% Change fonts: Section headers Sans-Serif:
\setsecheadstyle{\Large\sffamily\raggedright}
\setsubsecheadstyle{\large\sffamily\raggedright}
\setsubsubsecheadstyle{\normalsize\sffamily\raggedright}

% Title
\pretitle{\LARGE\sffamily}
\posttitle{\par\vspace{4ex}}
\preauthor{\large\sffamily\hspace{1cm}}
\postauthor{\par\vspace{3ex}}
\predate{\small\sffamily\hspace{1cm}Last Updated on: }
\postdate{\par\vspace{2cm}}
\copypagestyle{title}{plain}
\makeoddfoot{title}{}{}{}
\makeevenfoot{title}{}{}{}

\renewcommand{\abstractnamefont}{\sffamily}
\renewcommand{\abstracttextfont}{}
\renewcommand{\absnamepos}{flushleft}


\makechapterstyle{mestrado}{% Originally ``AlexanderGrebenkov'', adapted
\renewcommand{\chapterheadstart}{\goodbreak\vspace*{\beforechapskip}\medskip}
\renewcommand{\chapnamefont}{\normalfont\Large\scshape}
\renewcommand{\chapnumfont}{\normalfont\Large\scshape}
\renewcommand{\chaptitlefont}{\normalfont\Large\scshape}
\renewcommand{\printchaptername}{}
\renewcommand{\chapternamenum}{}
\renewcommand{\printchapternum}{\normalfont\Large\scshape\S\thechapter}
\renewcommand{\afterchapternum}{\hspace{1em}}
\renewcommand{\afterchaptertitle}{\par\nobreak\vspace{-.9em}\moveright 6pt\vbox to 1pt{\hrule width .64\textwidth}\nobreak\vskip\afterchapskip\nobreak}
}
\chapterstyle{mestrado}

\begin{document}

\maketitle

\begin{abstract}
In June and July 2010, Rita Reis and Luis Pedro Coelho, traveled to Mozambique
and taught theatre classes to high school students in four rural schools. Each
workshop was a week-long affair, with students working 1h30 per day.

Most of the students involved were already participating in school theatre
groups, which in the Mozambican context is closely linked to working as
volunteers for different organisations to spread information on topics such as
HIV/AIDS, other STDs, or education.

The classes were a mix of warm-up exercises for students (some of which formed
a core, repeated almost every day); text work, and improv work. In some cases,
there was interest in developing English language skills and some of the work
was done in English (some schools had English language theatre groups already,
organised by the Peace Corps\footnote{Peace Corps is a volunteering programme
of the American government.} volunteers).

The programme described here is the result of an evolution throughout our stay,
as we had the opportunity to work in different schools, with two groups each
time. We were able to modify the presentation and exercises themselves as we
perceive them to be more or less successful (there were, of course, other
modifications that were made in response to the specificity of each group). We
describe our experience here in the hope that it will be useful to someone else
in the future.

\end{abstract}

\chapter{Introduction}
In June and July 2010, we taught theatre classes to high school students in
four rural schools in Mozambique. These were run by EsMaBaMa, a non-profit
Catholic organisation and are located in Estaquinha, Machanga, Baranda, and
Mangunde (hence the name, EsMaBaMa), all in the Sofala province of Mozambique.

Theatre and poetry are an important part of Mozambican culture. We were
fortunate to be present for the celebration of Indenpendence Day
(June 25). The celebration in Machanga (a very small town) consisted of
speeches, live and recorded music, and several poetry readings. All of the
schools already had some theatre structure. Sometimes these were run by the
Peace Corps volunteers and performed in English in order to get the students to
practice that language.

We taught a one-week workshop in each mission, everytime having a morning and
an afternoon group (some of the students have morning class and others
afternoon classes, so they would attend our workshop when they were not in
class). In Estaquinha, at the request of the students and the school director,
the students prepared a small presentation at the end. Still, the programme
that we followed evolved throughout our stay as we observed and reflected on
the results of the previous experiments.

Our course had several goals: to support the existing theatre structures by
teaching them basic exercises they could continue using, to help support
Portuguese class by reiterating to the students some of the grammatical and
rhetorical concepts they learned, in some cases, to have the students practice
their English, and, finally, to make it fun for the students.

\chapter{Exercise Catalogue}
\section{Core}
\begin{itemize}
\item Spinal awakening,
\item spinal twists,
\item boom-sha-boom,
\item Oy, steady, take-it,
\item balance of the space.
\end{itemize}

\subsection{Spinal Awakening}
\subsection{Spinal Twists}
\subsection{Boom Sha Boom}
This was adapted in every school to include the name of the school, followed by
Sofala (the province) or Mozambique. The main text ``We've got the rythm in our
hands|feet|hips|eyes'' was sometimes translated into Portuguese, sometimes not,
depending on the general level of the students.

\begin{verse}
Ma-chan-ga, boom-sha-boom\\
Mo-zam-bi-que, boom-sha-boom\\
temos-o-rit-mo-nas-mãos/pés/ancas/olhos.
\end{verse}

Each time we used the name of the school, \textit{Ma-chan-ga},
\textit{Es-ta-qui-nha}, \textit{Ba-ra-da}, or \textit{Man-gun-de}.

\subsection{Oy, steady, take-it}
This was adapted into Portuguese, as \textit{oi-toma-espelho} (roughly,
\textit{Oy, have it, mirror}). An early attempt at using the English version
led to much confusion on part of the students.

\subsection{Balance of the Space}
Using the image of a plank balanced on a pole, the students attempt to
distribute themselves evenly across the space, while also following the
leaders' pace and movements. Initially the leader is the instructor who
occasionally stops, speeds up or slows down, or sits. Eventually, some of the
students can lead or other movements can be introduced.

The students should keep their gaze high.

\section{Improvs}

In some groups, improvs were in English (including at the request of the
students), sometimes it was in mixed language, sometimes we allowed them to use
an invented gibberish if they wished (but forbade Portuguese and
Ndau\footnote{Ndau is the local language. In truth, if the students chose to
speak Ndau, they would have been able to get away with it as neither of us
understood it.}). The students might not have always stuck to the rules, but
they did practice their English skills and some were much more fluent on stage
than when they were talking to the instructor.

Group improv settings were ``chapa'', health care centre, market, and party.

In Mozambique, a ``chapa'' is a 15-person minivan which serves as public
transportation carrying up to 20 passenger crammed into its seats. It is used
both in the city (where it is actually a very convenient way to get around as
the frequency is very high) and inter-city. The ticket collector, who also
calls out for passengers whenever the chapa stops (chapas stop on any
intersection as long as someone inside or outside calls out, but there are
semi-established stops), served as the leader in this improv. The stage started
with him alone on stage and he would call out to the rest of the group to join
him. When we had assembled a sufficient number of passengers, he'd call out for
a driver (this mimics the behaviour of actual chapas at terminal points when
the driver might step out for a little break as the ticket-collector gathers
new passengers). Other characters include salespersons, traffic police (who,
this being Mozambique, were willing for a bribe), the beggar, chapa owner, and,
in one inspired scene, the man ran over by the chapa. We describe it here at
length as it was invariably successful in terms of both including everyone,
being fun, and generating interesting scenes for the instructors watching.

The health care centre feature doctors, patients, receptionists, pharmacists,
and a helper. They were a mixed success. The market is a good improv for
continued action, but it runs the risk of boredom if many of the students just
come in, buy their bananas, and leave. The party improv was mostly interesting
for us as it revealed many of the typical teenage flirtation modes of
Mozambican teens (see discussion below, under \emph{Challenges}).

\section{Comedia Dell'Arte}

The traditional \textit{comedia dell'arte} form, which we will refer to as
simply \textit{comedia}, gave us many characters that are physically
well-defined and expansive.

\section{Group Text Work}


\subsection{Physicalise Verbs, Adjectives, or Important Words}

In a circle, each student will look for one verb and physicalise the
corresponding action. All of the other students will then imitate. We also
asked them to physicalise adjectives as a statue. Finally, we asked for
important keywords of the text which were physicalised.

\section{Individual Text Work}

Some of the exercises that were done as a group, can be repeated individually.
For example, physicaling verbs and adjectives can help the students find
appropriate action for their texts.

\subsection{Say it to a Partner}

Say the text to a partner on stage. This can be done in several variations:
face to face standing or sitting down, holding hands (this also helps to
contain small unmotivated hand gestures). The students can be close or far apart.

Another variation is to say the text, back-to-back with a partner.

\subsection{Push the Wall}

Push against the wall, exerting force. This can be done with the student
leaning against one wall or running from one wall to the opposite wall using
hands and arms to propel the body.

\subsection{Come Running from the River}

We asked the student to either run circling the room or to actually leave the
room, walk or run to a nearby and tree, then run back, barge in an say the poem
as if they had ran all the way from the river with this important message. This
should stimulate their feelings of urgency.

\section{Other Exercises}
\subsection{Object Transformation}
An object (such as a pencil-case) was passed around and pupils were asked to
make it as if it was something else (e.g., a mirror, a shoe,\dots). In some of
the classes, the remaining students would call out loud the name of the object
in English. This thus served as an exercise that developed vocabulary, imagination,
and the ability to play in the moment.

\subsection{Voice and Movement Creatures}
\subsection{Circling Clap}
In a circle, student one and student two clap their hands simmulataneously,
then student two and student three, and so forth. If the students clap twice,
then there is a direction change.

\chapter{Workshop Structure}

We didn't follow a strict schedule and the work varied. We document here one
possible structure as a reference.

\section{Day I}
The first day is an introduction. We wish to introduce both some of the core
exercises and have the students get a feel for some of the work methods. We
also attempted to make it a fun day in order to keep the students interested.
Thus, we rewarded the students with an improv at the end of the first session.

\begin{itemize}
\item Warm-up
\item Oi, toma, espelho,
\item Boom-sha-boom,
\item Circling clap,
\item Object transformation,
\item Imitate thy neighbour,
\item Physicalise your name or your name and verb,
\item Objective: chair,
\item Improv: chapa.
\end{itemize}
\section{Day II}

We repeat some of the core exercises.

\begin{itemize}
\item Oi, toma, espelho,
\item Boom-sha-boom,
\item Circling clap,
\item Physicalise verbs,
\item Poem distribution,
\item Read poem whispering, soto voce, aloud; stopped or walking,
\item Back-to-back, single out important words,
\item Machines,
\item Balance of the space.
\end{itemize}
\section{Day III}
In this session, after the warm-up, we do some \textit{comedia} characters in
the circle or moving around the room, with everyone moving together. We use
those characters for some \textit{comedia} improv work. At the end of the
session, we start some individual text work with those students who feel they
have memorised their texts. Finally, to end the class on a fun note which
brings everyone together after individual work, we do boom-sha-boom.

\begin{itemize}
\item Oi, toma, espelho,
\item circling clap,
\item spinal awakening,
\item \textit{comedia} characters, in a circle,
\item \textit{comedia} improvs,
\item individual text work,
\item Boom-sha-boom.
\end{itemize}
\section{Day IV}

We continue with the themes from the previous day. We go over figures of speech
concepts and ask the students to identify some in their texts in a group. We
end with individual text work. By now, we expect of the students that they all
know their texts (which was, of course, not always the case).

\begin{itemize}
\item Circling clap,
\item spinal awakening with partner,
\item balance of the space with text,
\item figures of speech analysis,
\item \textit{comedia} characters,
\item physicalise sentence from text, in a circle,
\item individual text work.
\end{itemize}
\section{Day V}

This day is mostly dedicated to individual work with students presenting poems.
We tried to make sure that everybody had a chance to go at least once, while
some students went more than once.

\begin{itemize}
\item Circling clap,
\item spinal twists,
\item zoo-aw-sha,
\item balance of the space with different leaders,
\item improv: sustained actions in pairs, then interaction,
\item individual text work.
\end{itemize}

\chapter{Challenges}
\section{Language}

Mozambique is home to many languages. In the region of Sofala where we were
working, Ndau is the most common language (Sena is also spoken in Sofala).
Portuguese is the official language of the country, but, particularly in the
country-side, some students only encounter it when they reach school age.
Therefore, their levels vary widely and some students struggle with vocabulary.

\section{Structured Work}

Given the voluntary nature of the work and Mozambican attitudes, students were
often late. We attempted to always book a two hour slot, but keep the session
to exactly ninety minutes from the moment we actually started. We informed the
students of this fact. It was very helpful if one of their teachers was
present, but this was not always possible. We decided to start when we felt we
had enough of a critical mass to avoid waiting too long or not being able to
end on time, but this meant that there were students trickling in for a long
time.

Similarly, it often happened that a student would wish to join the group
half-way or even at the last session. While we enjoyed the appreciation that
this represented (as opposed to the more typical waning of enthusiasm and
attendance, in some cases, total attendance went up) We tried to always
accommodate them, but it sometimes slowed progress.

\section{Women's Position in Mozambican Society}

Mozambican society is highly patriarchal. This shows up both in the content
of the work (see below), but also in the way that women worked. In some groups,
there were very charismatic, talented female students, of course; but, on
average, the female students tended to be less self-confident and come forward
less.

In one occasion, a female student volunteered to play ticket-collector in the
``chapa'' improv, to which some of the male students reacted negatively saying
that a ``woman cannot be ticket-collector.'' This discouraged the girl and even
after we said that she could play the role, she was no longer interested.

\section{Mozambican Societal Issues}

Mozambican society had many problems which showed up again and again the
improvisation work. To some extent, this is the result of the role that theatre
plays in Mozambican society, very often focused on social issues; but it is
also a reflection of a society with many grave problems.

The corrupt traffic policeman was a staple of the chapa improv. In some cases,
he would ask for money directly, other times,the driver would offer it.

The women's position and relationship to sexuality and money was also
repeatedly portrayed as one of business. In the ``party'' improvisation,
seduction was often around the direct statement of ``I'm rich, I can buy you
presents'' (or conversely, ``I'm not interested. I already have a boyfriend and
he's rich.'').\footnote{Another interesting exchange took place in a 3-person
improv we directed in another context. The male element in a couple tries to
bribe a policeman with a small bill, saying ``it's all the money I have;'' to
which the policeman replies by turning to the girl and asking ``how can you
date a guy who has no money? He just tried to give me a ridiculous bribe!''}
This pattern showed up even when we asked the students to play switched
genders. In another case, a woman was prostituted in the market with one of the
male students playing an intermediary who took a cut. These are all issues that
are present in Mozambican society, with many relationships between men and
women straddling the line between gift offering and outright prostitution

HIV/AIDS was present too. Not only where it was obvious (health care centre),
but everywhere a condom salesman would show up, for example, and one ``market''
improv included one thorough explanation of how to use a condom\footnote{In
this case, we actually appropriated this scene and re-fashioned it for the
``chapa'' improv, where the salesman would approach the bus as it stopped. This
was part of the presentation shown to the school.}). The students were all
very knowledgeable of the issues and, in a way that in the West would be very
surprising for their age, able to discuss it in front of others in a completely
matter-of-fact way. One improv, which we set up as ``sexual education class''
was reminiscent of that Monty Python sketch where the teacher, played by John
Cleese, explains sex to a bored audience who is thoroughly uninterested. In our
case, the students simply answered the teacher's questions as good students
would answer questions on History or some other subject.

Sometimes the students reacted against an overly negative portrayal of
Mozambique with the argument that they shouldn't only show the problems in
society but also that there are good things too. This led us to explore
alternative choices with the characters. In Estaquinha, we played a repeated
scene in which three friends meet after taking an HIV-test. One of them is
positive and is alternatively rejected by his ignorant friends (``that's why
you have those pimples! Even those ripped trousers are part of the disease! I
can't talk to you! We'll get contaminated.'') or embraced and correctly
counselled by them (``Don't look that discouraged! You'll outlive us all! Just
take your medication. And we'll still be your friends, come, let's have a
drink!''). These two scenes were worked into two improvs that were presented to
the school one after the other.

\chapter{Assets}
As well as a unique set of challenges, Mozambican students had many assets when
compared to Western students of the same age.

Theatre is very present in Mozambican culture and is very valued. Many of our
students were \textit{activistas} in various youth groups. While in European
Portuguese the word \textit{activista} has the same connotation as the English
word \textit{activist}, in Mozambique, the expression has wider usage. In fact,
there is widespread youth social participation in organised groups such as
\textit{JOMA}, Youth for Change and Action,\footnote{\textit{Jovens Para
Mudança e Acção}, in Portuguese.} which organises around HIV related issues.
Participation in these groups is taken seriously and is a source of pride.

The students were freer with their bodily expression. They had less social
inhibitions to moving freely.  The male students were not afraid to play female
or effeminate characters.

As a counterpart to the lack of rigour in preparation, the students are
unafraid of improvising both in class and in front of the audience.\footnote{In
Estaquinha, when we presented in front of an audience, we were both laughing at
the new jokes that the students came up with on the spot.} 

\chapter{Evolution}

We evolved the programme to introduce improvs from the start. Similarly,
\textsl{comedia} exercises were found to work very well once we tried them and,
then, became part of the basic structure.

\chapter{Conclusions}

We taught this class to over eighty students in four different rural Mozambican
schools.

We felt that the classes were genuinely helpful for the students. We were
fortunate that we were able to teach this in the context of existing groups of
\textit{activistas} involved in theatre and, therefore, some of the work served
as support to an existing structure.

\end{document}
