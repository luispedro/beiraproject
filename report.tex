\documentclass[article,twocolumn,twoside]{memoir}
\usepackage[utf8]{inputenc}
\usepackage{psfrag}
\usepackage{hyperref}
\usepackage{color}
\usepackage[]{graphicx}

\graphicspath{{../figures/other/}{../figures/generated/}}


\title{Theatre for Development: Theatre Classes in Rural Mozambique}
\author{Rita Reis and Luis Pedro Coelho}

\bibliographystyle{alpha}
%\bibliographystyle{plain}
\pagestyle{Ruled}

\aliaspagestyle{chapter}{Ruled}
\makeatletter
\if@twoside
	\makeoddhead{Ruled}{}{}{\scshape\rightmark}
	\makeevenhead{Ruled}{\scshape\leftmark}{}{}
	\makeevenfoot{Ruled}{\pagenumberfont\thepage}{}{}
	\makeoddfoot{Ruled}{}{}{\pagenumberfont\thepage}
	% Put section number on top
	\def\sectionmark#1{\markright{#1 (\thesection)}}
	\def\chaptermark#1{\markboth{#1}{#1}}
	\renewcommand*{\bibmark}{\markboth{\bibname}{\bibname}} % I don't like empty headings!
\else

	% Put section number on top
	\def\sectionmark#1{\markright{#1 (\thesection)}}
	\def\chaptermark#1{\markright{#1 (\thechapter)}}

	\makeoddhead{Ruled}{}{}{\scshape\rightmark}
	\makeevenhead{Ruled}{}{}{\scshape\rightmark}
	\makeevenfoot{Ruled}{}{}{\pagenumberfont\thepage}
	\makeoddfoot{Ruled}{}{}{\pagenumberfont\thepage}
	\renewcommand*{\bibmark}{\markright{\bibname}} % I don't like empty headings!


\fi
\makeatother

%Change fonts: Page number sans-serif (much cleaner than the roman version)
\def\pagenumberfont{\sffamily}
% Change fonts: Section headers Sans-Serif:
\setsecheadstyle{\Large\sffamily\raggedright}
\setsubsecheadstyle{\large\sffamily\raggedright}
\setsubsubsecheadstyle{\normalsize\sffamily\raggedright}

% Title
\pretitle{\LARGE\sffamily}
\posttitle{\par\vspace{4ex}}
\preauthor{\large\sffamily\hspace{1cm}}
\postauthor{\par\vspace{3ex}}
\predate{\small\sffamily\hspace{1cm}Last Updated on: }
\postdate{\par\vspace{2cm}}
\copypagestyle{title}{plain}
\makeoddfoot{title}{}{}{}
\makeevenfoot{title}{}{}{}

\renewcommand{\abstractnamefont}{\sffamily}
\renewcommand{\abstracttextfont}{}
\renewcommand{\absnamepos}{flushleft}


\makechapterstyle{mestrado}{% Originally ``AlexanderGrebenkov'', adapted
\renewcommand{\chapterheadstart}{\goodbreak\vspace*{\beforechapskip}}
\renewcommand{\chapnamefont}{\normalfont\Large\scshape}
\renewcommand{\chapnumfont}{\normalfont\Large\scshape}
\renewcommand{\chaptitlefont}{\normalfont\Large\scshape}
\renewcommand{\printchaptername}{}
\renewcommand{\chapternamenum}{}
\renewcommand{\printchapternum}{\normalfont\Large\scshape\S\thechapter}
\renewcommand{\afterchapternum}{\hspace{1em}}
\renewcommand{\afterchaptertitle}{\par\nobreak\vspace{-.9em}\moveright 6pt\vbox to 1pt{\hrule width .32\textwidth}\nobreak\vskip\afterchapskip\nobreak}
}
\chapterstyle{mestrado}

\begin{document}

\maketitle

\begin{abstract}
In June and July 2010, Rita Reis and Luis Pedro Coelho, traveled to Mozambique
and taught theatre classes to high school students in four rural schools. Each
workshop was a week-long affair, with students working 1h30 per day.

Most of the students involved were already participating in school theatre
groups, which in the Mozambican context is closely linked to working as
volunteers for different organisations to spread information on topics such as
HIV/AIDS, other STDs, or education.

The classes were a mix of warm-up exercises for students (some of which formed
a core, repeated almost every day), text work, and improv work. In some cases,
there was interest in developing English language skills and some of the work
was done in English (some schools had English language theatre groups already,
organised by the Peace Corps\footnote{Peace Corps is a volunteering programme
of the American government.} volunteers).

The programme described here is the result of an evolution throughout our stay,
as we had the opportunity to work in different schools, with two groups each
time. We were able to modify the presentation and exercises themselves as we
perceive them to be more or less successful (there were, of course, other
modifications that were made in response to the specificity of each group). We
describe our experience here in the hope that it will be useful to someone else
in the future.

\end{abstract}

\chapter{Introduction}
In June and July 2010, we taught theatre classes to high school students in
four rural schools in Mozambique. These were run by EsMaBaMa, a non-profit
Catholic organisation and are located in Estaquinha, Machanga, Baranda, and
Mangunde (hence the name, EsMaBaMa), all in the Sofala province of Mozambique.

All of the schools already had some theatre structures. Sometimes these were
run by the Peace Corps volunteers and performed in English in order to get the
students to practice that language.

We taught a one-week workshop in each mission, everytime having a morning and
an afternoon group (some of the students have morning class and others
afternoon classes, so they would attend our workshop when they were not in
class). In Estaquinha, at the request of the students and the school director,
the students prepared a small presentation at the end.

Our course had several goals: to support the existing theatre structures by
teaching them basic exercises they could continue using, to help support
Portuguese class by reiterating to the students some of the grammatical and
rhetorical concepts they learned, in some cases, to have the students practice
their English, and, finally, to make it fun for the students.

The programme that we followed evolved throughout our stay as we observed and
reflected on the results of the previous experiments. We present the programme
as it resulted from our experience, but we comment on the evolution of it
below.

\chapter{Workshop Structure}

\section{Core}

We decided to have a small set of \textit{core} exercises that we repeated
everyday, so that they would be remembered after we left. Other exercises were
used only once or as-needed. This was a mix of warm-up and group concentration
exercises.

\subsection{Spinal Awakening}
\subsection{Spinal Twists}
\subsection{Zoo-Aw-Sha}
\subsection{Boom Sha Boom}

This is a group song and rythmic routine, performed in a circle. The basic
movement is to (1) clap your hands on your tighs, (2) clap them, (3) clap with
your neighbours to each side, and (4) clap; resulting in a four-tempo beat.
Originally, the words were ``Yu-gos-la-via, boom-sha-boom \textit{clap},
Che-kos-lo-vaquia, boom-sha-boom \textit{clap}'' followed by ``We've got the
rhythm in our hands'' (with clapping of hands), our feet (with stomping), our
hips (jiggling the hips to ``hu-hu-hu'') and eyes (moving your hands from yours
eyes to the centre of the group, followed by the whole body to ``uuu-huuu'')

Instead of defunct European countries, we used the name of the school,
\textit{Ma-chan-ga}, \textit{Es-ta-qui-nha}, \textit{Ba-ra-da}, or
\textit{Man-gun-de} followed by \textit{So-fa-la} or \textit{Mo-zam-bi-que}.
The main text ``We've got the rhythm in our hands|feet|hips|eyes'' was
sometimes translated into Portuguese,\footnote{Temos o ritmo nas
mãos|pés|ancas|olhos.} sometimes not, depending on the general level of the
students.

\begin{verse}
Ma-chan-ga, boom-sha-boom\\
Mo-zam-bi-que, boom-sha-boom\\
temos-o-rit-mo-nas-mãos/pés/ancas/olhos.
\end{verse}


\subsection{Oy, steady, take-it}
In a circle, there are three ways in which energy can circulate: (1)
\textit{Oy} passes it to the person next to you, (2) \textit{steady} passes it
across the circle to someone else, and (3) \textit{take-it} gives it back to
whom it came from. The participants should use these words and corresponding
hand-gestures. The goals of the exercise is to have group concentration and eye
contact between the partners.

This was adapted into Portuguese, as \textit{oi, toma, espelho} (roughly,
\textit{Oy, have it, mirror}). An early attempt at using the English version
led to much confusion on part of the students. We introduced the game
gradually, starting with \textit{Oy}, then \textit{toma}, and finally
\textit{espelho}.

\subsection{Balance of the Space}

Using the image of a plank balanced on a pole below the centre of the room, the
students attempt to distribute themselves evenly across the space, while also
following the pace and movements of the leader. Initially the leader is the
instructor who occasionally stops, speeds up or slows down, or sits.
Eventually, some of the students can lead or other movements can be introduced.

The students should keep their gaze high and face forward. This is to train the
actor to be aware of the whole of his surroundings even when facing forward.

This is an exercise that the students get quickly and enjoy. When they play the
leader, they would (as the instructors did) stop, causing everyone to stop, and
then comment on the distribution of everyone else (only the leader should
speak, everyone else should keep quiet).

\section{Improvs}

In some groups, improvs were in English (including at the request of the
students), sometimes it was in mixed language, sometimes we allowed them to use
an invented gibberish if they wished (but forbade Portuguese and
Ndau\footnote{Ndau is the local language. Actually, if the students chose to
speak Ndau, they would have been able to get away with it as neither of us
understood it.}). The students might not have always stuck to the rules, but
they did practice their English skills and some were much more fluent on stage
than when they were talking to the instructor.

Group improv settings were ``chapa'', health care centre, market, and party.

In Mozambique, a ``chapa'' is a 15-person minivan which serves as public
transportation carrying up to 20 passenger crammed into its seats. It is used
both in the city\footnote{In the city, the chapas are actually a very
convenient way to get around as the frequency is very high.} and inter-city.
The ticket collector, who also calls out for passengers whenever the chapa
stops (chapas stop on any intersection as long as someone inside or outside
calls out, but there are semi-established stops), served as the leader in this
improv. The stage started with him alone on stage and he would call out to the
rest of the group to join him. When we had assembled a sufficient number of
passengers, he'd call out for a driver (this mimics the behaviour of actual
chapas at terminal points when the driver might step out for a little break as
the ticket-collector gathers new passengers). Other characters include
salespersons, traffic police (who, this being Mozambique, were willing for a
bribe), the beggar, chapa owner, and, in one inspired scene, the man ran over
by the chapa. We describe it here at length as it was invariably successful in
terms of both including everyone, being fun, and generating interesting scenes
for the instructors watching. Therefore, we used it as the first improv.

The health care centre feature doctors, patients, receptionists, pharmacists,
and a helper. They were a mixed success. The market is a good improv for
continued action, but it runs the risk of boredom if many of the students just
come in, buy their bananas, and leave. The party improv was mostly interesting
for us as it revealed many of the typical teenage flirtation modes of
Mozambican teens (see discussion below, under \emph{Challenges}).

\section{Comedia Dell'Arte}

The traditional \textit{comedia dell'arte} form, which we will refer to as
simply \textit{comedia}, gave us many characters that are physically
well-defined and expansive.

We worked with \textit{comedia} in two ways. First, we worked on the characters
with all students, in a circle or walking in the space. We had a few
characteristic sentences and movements that students were to repeat altogether.

After we had introduced the characters, we used some of the \textit{comedia}
characters and settings for improv. In particular, the \textit{capitano} meets
the \textit{signora}, she attempting to get her to marry her\footnote{or have
unprotected sex with her, in one of our adaptations.}, he attempting to get a
meal while pretending to be rich.\footnote{The \textit{capitano} lost his
fortune, but refuses to admit it. Therefore, he must be boastful while living
off others. The \textit{signora} is an older lady with some money who is
desperate for marriage.}

\section{Group Text Work}

In order to help the students first memorise their texts and then understand
and interpret them, we led a series of exercises that could be performed
together by the group.

\subsection{Physicalise Verbs, Adjectives, or Important Words}

In a circle, each student will look for one verb and physicalise the
corresponding action. All of the other students will then imitate. We also
asked them to physicalise adjectives as a statue. Finally, we asked for
important keywords of the text which were physicalised.

\section{Individual Text Work: Poetry}

As texts, we chose Portuguese-language poems from African poets that were part
of the students' high-school curriculum. We often selected only a few stanzas
which we asked the students to memorise (or broke the poem into two parts,
assigning each to a different student).

A large part of our workshop was dedicated to this work of finding a form with
which to communicate the poem.

We also used to poems as a basis for reviewing figures of speech and then
asking students to identify some in their texts.

Some of the exercises that were done as a group, can be repeated individually.
For example, physicaling verbs and adjectives can help the students find
appropriate action for their texts.

\subsection{Say it to a Partner}

Say the text to a partner on stage. This can be done in several variations:
face to face standing or sitting down, holding hands (this also helps to
contain small unmotivated hand gestures). The students can be close or far apart.

Another variation is to say the text, back-to-back with a partner.

\subsection{Push the Wall}

Push against the wall, exerting force. This can be done with the student
leaning against one wall or running from one wall to the opposite wall using
hands and arms to propel the body.

\subsection{Come Running from the River}

We asked the student to either run circling the room or to actually leave the
room, walk or run to a nearby and tree, then run back, barge in an say the poem
as if they had ran all the way from the river with this important message. This
should stimulate their feelings of urgency.

\section{Other Exercises}
\subsection{Object Transformation}
An object (such as a pencil-case) was passed around and pupils were asked to
make it as if it was something else (e.g., a mirror, a shoe,\dots). In some of
the classes, the remaining students would call out loud the name of the object
in English. This thus served as an exercise that developed vocabulary, imagination,
and the ability to play in the moment.

\subsection{Voice and Movement Creatures}
\subsection{Circling Clap}
In a circle, student one and student two clap their hands simmulataneously,
then student two and student three, and so forth. If the students clap twice,
then there is a direction change.

\chapter{Daily Schedule}

We didn't follow a strict schedule and the work varied both as our ideas
evolved and as we adapted it to each group. We document here one possible
structure as a reference.

\section{Day I}
The first day is an introduction. We wish to introduce both some of the core
exercises and have the students get a feel for some of the work methods. We
also attempted to make it a fun day in order to keep the students interested.
Thus, we rewarded the students with an improv at the end of the first session.

\begin{itemize}
\item Warm-up
\item Oi, toma, espelho,
\item Boom-sha-boom,
\item Circling clap,
\item Object transformation,
\item Imitate thy neighbour,
\item Physicalise your name or your name and verb,
\item Objective: chair,
\item Improv: chapa.
\end{itemize}
\section{Day II}

We repeat some of the core exercises. We introduce the poems for the first
time. Students who did not bring a poem of their own, the majority, got a
randomly chosen one from us.

\begin{itemize}
\item Oi, toma, espelho,
\item Boom-sha-boom,
\item Circling clap,
\item Physicalise verbs,
\item Poem distribution,
\item Read poem whispering, soto voce, aloud; stopped or walking,
\item Back-to-back, single out important words,
\item Machines,
\item Balance of the space.
\end{itemize}
\section{Day III}
In this session, after the warm-up, we do some \textit{comedia} characters in
the circle or moving around the room, with everyone moving together. We use
those characters for some \textit{comedia} improv work. At the end of the
session, we start some individual text work with those students who feel they
have memorised their texts. Finally, to end the class on a fun note which
brings everyone together after individual work, we do boom-sha-boom.

\begin{itemize}
\item Oi, toma, espelho,
\item circling clap,
\item spinal awakening,
\item \textit{comedia} characters, in a circle,
\item \textit{comedia} improvs,
\item individual text work,
\item Boom-sha-boom.
\end{itemize}
\section{Day IV}

We continue with the themes from the previous day. We go over figures of speech
concepts and ask the students to identify some in their texts in a group. We
end with individual text work. By now, we expect of the students that they all
know their texts (which was, of course, not always the case).

\begin{itemize}
\item Circling clap,
\item spinal awakening with partner,
\item balance of the space with text,
\item figures of speech analysis,
\item \textit{comedia} characters,
\item physicalise sentence from text, in a circle,
\item individual text work.
\end{itemize}
\section{Day V}

This day is mostly dedicated to individual work with students presenting poems.
We tried to make sure that everybody had a chance to go at least once, while
some students went more than once.

\begin{itemize}
\item Circling clap,
\item spinal twists,
\item zoo-aw-sha,
\item balance of the space with different leaders,
\item improv: sustained actions in pairs, then interaction,
\item individual text work.
\end{itemize}

\chapter{Challenges}
\section{Language}

Mozambique is home to many languages. In the region of Sofala where we were
working, Ndau is the most common language (Sena is also spoken in Sofala).
Portuguese is the official language of the country, but, particularly in the
country-side, some students only encounter it when they reach school age. In
the city, Portuguese is current, and sometimes a first language. Given that
some of our students came from the country side, others from the city, and from
different background, their language skills varied widely and some students
struggled with vocabulary.

\section{Structured Work}

Given the voluntary nature of the work and Mozambican attitudes, students were
often late. We attempted to always book a two hour slot, but keep the session
to exactly ninety minutes from the moment we actually started. We informed the
students of this fact. It was very helpful if one of their teachers was
present, but this was not always possible. We decided to start when we felt we
had enough of a critical mass to avoid waiting too long or not being able to
end on time, but this meant that there were students trickling in for a long
time.

Similarly, it often happened that a student would wish to join the group
half-way or even at the last session. We enjoyed the appreciation that this
represented (as opposed to the more typical waning of enthusiasm and
attendance, in some cases, total attendance went up) and we tried to always
accommodate them, but it sometimes slowed progress.

\section{Women's Position in Mozambican Society}

Mozambican society is highly patriarchal. This shows up both in the content
of the work (see below), but also in the way that women worked. In some groups,
there were very charismatic, talented female students, of course; but, on
average, the female students tended to be less self-confident and come forward
less.

In one occasion, a female student volunteered to play ticket-collector in the
``chapa'' improv, to which some of the male students reacted negatively saying
that a ``woman cannot be ticket-collector.'' This discouraged the girl and even
after we said that she could play the role, she was no longer interested.

In general, female students, perhaps internalising these relationships were
less forthcoming than male students when we asked for volunteers or for someone
who was ready to present their work. All we could do was to then insist on
having female volunteers as well as male one.

\section{Mozambican Societal Issues}

Mozambican society had many problems which showed up again and again the
improvisation work. To some extent, this is the result of the role that theatre
plays in Mozambican society, very often focused on social issues; but it is
also a reflection of a society with many grave problems.

The corrupt traffic policeman was a staple of the chapa improv. In some cases,
he would ask for money directly; other times, the driver would offer it. We
sometimes suggested that a policeman would show up, but we never suggested he
would be corrupt---that the students introduced themselves. In fact, a traffic
policeman asking for documents is generally assumed to be looking for
bribes.\footnote{In one occasion, while in the city, the chapa we were riding
was pulled over by a policeman asking for documents. We watched, with interest,
having seen this scene played out by the students in improv. Sure enough, the
policeman kept the driver's license, the ticket-collector got out a few yards
further, and ran back to get the documents (certainly, not for free). Even if
anecdotally, we can report that traffic policeman do look for bribes in
Mozambique.}

The women's position and relationship to sexuality and money was also
repeatedly portrayed as one of business. In the ``party'' improvisation,
seduction was often around the direct statement of ``I'm rich, I can buy you
presents'' (or conversely, the female refusal was of the form ``I'm not
interested. I already have a boyfriend and he's rich.'').\footnote{Another
interesting exchange took place in a 3-person improv we directed in another
context. The male element in a couple tries to bribe a policeman with a small
bill, saying ``it's all the money I have;'' to which the policeman replies by
turning to the girl and asking ``how can you date a guy who has no money? He
just tried to give me a ridiculous bribe!''} This pattern showed up even when
we asked the students to play switched genders. In another case, a woman was
prostituted in the market with one of the male students playing an intermediary
who promised to take a small cut, but then took all of the money for himself.
These are all issues that are present in Mozambican society, with many
relationships between men and women straddling the line between gift offering
and outright prostitution

HIV/AIDS was present too. Not only where it was obvious (health care centre),
but everywhere a condom salesman would show up, for example, and one ``market''
improv included one thorough explanation of how to use a condom.\footnote{In
this case, we actually appropriated this scene and re-fashioned it for the
``chapa'' improv, where the salesman would approach the bus as it stopped. This
was part of the presentation shown to the school.} The students were all
very knowledgeable of the issues and, in a way that in the West would be very
surprising for their age, able to discuss it in front of others in a completely
matter-of-fact way. One improv, which we set up as ``sexual education class''
was reminiscent of that Monty Python sketch where the teacher, played by John
Cleese, explains sex to a bored teenage audience who is as thoroughly
uninterested as they would be in lessons in Ancient Greek. In our case, the
students simply answered the teacher's questions as good students would answer
questions on History or some other subject. There was little tension in this
improv, dramatic or otherwise, and we did not use it again.

Sometimes the students reacted against an overly negative portrayal of
Mozambique with the argument that they shouldn't only show the problems in
society but also that there are good things too. This led us to explore
alternative choices with the characters. In Estaquinha, we played a repeated
scene in which three friends meet after taking an HIV-test. One of them is
positive and is alternatively rejected by his ignorant friends (``that's why
you have those pimples! Even those ripped trousers are part of the disease! I
can't talk to you! We'll get contaminated.'') or embraced and correctly
counselled by them (``Don't look that discouraged! You'll outlive us all! Just
take your medication. And we'll still be your friends, come, let's have a
drink!''). These two scenes were worked into two improvs that were presented to
the school one after the other.

\chapter{Assets}
As well as a unique set of challenges, Mozambican students had many assets when
compared to Western students of the same age.

Theatre and poetry are an important part of Mozambican culture. We were
fortunate to be present for the celebration of Indenpendence Day (June 25). The
celebration in Machanga (a very small town) consisted of speeches, live and
recorded music, and several poetry readings.

Many of our students were \textit{activistas} in various youth groups. While in
European Portuguese the word \textit{activista} has the same connotation as the
English word \textit{activist}, in Mozambique, the expression has wider usage
to denote organised volunteering. In fact, there is widespread youth social
participation in organised groups such as \textit{JOMA}, Youth for Change and
Action,\footnote{\textit{Jovens Para Mudança e Acção}, in Portuguese.} which
organises around HIV related issues and was present in every school we visited.
Participation in these groups is taken seriously and is a source of pride for
the \textit{activistas}. This seemed to us to be a feature of Mozambican
society and not only for younger students---older people also took pride in
participating in civil society activities.

There is also a national competition on English language theatre for high
school students, which is further motivation that students have for
participating in theatre groups. Therefore, our work was shaped to serve these
existing structures in their activities rather than to build up new structures.
The teachers who organise these groups were also invariably enthusiastic and an
indispensible help in handling logistic details and interfacing with the
students and school.

At each location, there was always an enthusiastic core of students who
participated and were committed. They took responsibility for the rest of the
group and were often naturally in the role of serving as an interface between
us, the other students, and the school---which was invaluable when their
teacher could not be present and we did not know whom else to approach for some
practical issue.

On stage, the students were freer with their bodily expression and less
self-conscious than we feel Western students of the same age and theatrical
experience are. They had less social inhibitions to moving freely with their
whole bodies. We also remarked that the male students were not afraid to play
female or effeminate characters.

As a counterpart to the lack of rigour in preparation, the students are
very capable of improvising both in class and in front of the
audience.\footnote{In Estaquinha, when we presented in front of an audience, we
were both laughing at the new jokes that the students came up with on the
spot.} They also did not feel stumped by a lack of rigorous preparation before
presenting (which, due to time constraints, we did not have).

The theatre groups run by the students generally work with ``prepared
improvs.'' A scene will start out as an improvisation based on some general
theme or set of characters, they will pick the best lines, rehearse it a few
times (but not more), until the general lines of action and dialogue become
clear, and then present. Still, even when presenting, the text is not
completely set, and is never written down anywhere. Therefore, they have
experience and an intuitive understanding of this type of work and it was
generally very good.

\chapter{Evolution}

The programme changed as we gained experience and we have, so far, mostly
described our final ideas. Some exercises were found to work better than others
and our thinking changed. We narrowed down on a smaller set of exercises than
we started with under the reasoning that it was best to repeat the same
exercise several times so that the existing theatre groups could keep a memory
of that exercise and use it in the future.

We found that the students loved doing improv work and were immediately
good at it. Therefore, we moved it to the first session in order to keep them
motivated. Similarly, \textit{comedia} characters worked very well as did
\textit{comedia}-based improvisation.

We hope that the students will be able to use \textit{comedia} characters (or,
at least, \textit{comedia}-inspired characters) in their further work.

Initially, we had imagined dramatising traditional Mozambican folk tales in
parallel or as an alternative to the poems. We rapidily dropped idea as very
few of the students were volunteering stories of their own (we had given them
the choice of picking it out of a book we had identified at the school library
or bringing something someone had told them as children). Furthermore, our
fears that the students would be resistant to poetry and would strongly prefer
narrative prose were unfounded. Given the time limitations, we chose to focus
only on the poetry.

\chapter{Conclusions}

We taught this class to over 150~students in four different rural Mozambican
schools. The class evolved as we experimented different ideas and exercises.

We were fortunate that we were able to teach this in the context of existing
groups of \textit{activistas} involved in theatre and, therefore, some of the
work served as support to an existing structure.

Knowing the common language of the country was essential. Even the English
theatre groups were run in Portuguese with only the performances itself being
in English. When describing an improv which was to be in English, we used
English to get the group to start thinking in English, but were sometimes
forced to translate at least some of the words.

On one occasion, the students presented to the school community, a very tough
audience, who would laugh at a student's mishaps. The students varied in their
preparedness, given that some had not even attended all of the sessions. We
gave everyone an opportunity if they felt that they were ready, but we tried to
give them a longer or shorter time on stage depending to avoid, as much as
possible, that they would feel embarassed. We had previously worried that our
intuitions of what is a good performance would not match what a Mozambican
audience would enjoy, but we were heartened to find that the audience agreed
with us as to which of the performances were best.

Poetry was a good source of text material for the work. The students enjoyed it
and worked hard on it. We had, as an alternative, the idea of dramatising
traditional folk tales, under the assumption that a narrative text would be
better; but we never needed to resort to those texts. Poems, being short, are
easier for students to memorise, and allowed for individual text work, which
will be useful for any public speaking situation that they might face. It also
allowed us to sneak in some reviews of rethorical concepts and, sometimes, our
sessions resembled Portuguese class.

\end{document}
