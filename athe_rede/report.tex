\documentclass[article,twocolumn,twoside]{memoir}
\usepackage[utf8]{inputenc}
\usepackage{psfrag}
\usepackage{hyperref}
\usepackage{color}
\usepackage[]{graphicx}

\graphicspath{{../figures/other/}{../figures/generated/}}


\title{Theatre for Development: Theatre Classes in Rural Mozambique}
\author{Rita Reis and Luis Pedro Coelho}

\bibliographystyle{alpha}
%\bibliographystyle{plain}
\pagestyle{Ruled}

\aliaspagestyle{chapter}{Ruled}
\makeatletter
\if@twoside
	\makeoddhead{Ruled}{}{}{\scshape\rightmark}
	\makeevenhead{Ruled}{\scshape\leftmark}{}{}
	\makeevenfoot{Ruled}{\pagenumberfont\thepage}{}{}
	\makeoddfoot{Ruled}{}{}{\pagenumberfont\thepage}
	% Put section number on top
	\def\sectionmark#1{\markright{#1 (\thesection)}}
	\def\chaptermark#1{\markboth{#1}{#1}}
	\renewcommand*{\bibmark}{\markboth{\bibname}{\bibname}} % I don't like empty headings!
\else

	% Put section number on top
	\def\sectionmark#1{\markright{#1 (\thesection)}}
	\def\chaptermark#1{\markright{#1 (\thechapter)}}

	\makeoddhead{Ruled}{}{}{\scshape\rightmark}
	\makeevenhead{Ruled}{}{}{\scshape\rightmark}
	\makeevenfoot{Ruled}{}{}{\pagenumberfont\thepage}
	\makeoddfoot{Ruled}{}{}{\pagenumberfont\thepage}
	\renewcommand*{\bibmark}{\markright{\bibname}} % I don't like empty headings!


\fi
\makeatother

%Change fonts: Page number sans-serif (much cleaner than the roman version)
\def\pagenumberfont{\sffamily}
% Change fonts: Section headers Sans-Serif:
\setsecheadstyle{\Large\sffamily\raggedright}
\setsubsecheadstyle{\large\sffamily\raggedright}
\setsubsubsecheadstyle{\normalsize\sffamily\raggedright}

% Title
\pretitle{\LARGE\sffamily}
\posttitle{\par\vspace{4ex}}
\preauthor{\large\sffamily\hspace{1cm}}
\postauthor{\par\vspace{3ex}}
\predate{\small\sffamily\hspace{1cm}Last Updated on: }
\postdate{\par\vspace{2cm}}
\copypagestyle{title}{plain}
\makeoddfoot{title}{}{}{}
\makeevenfoot{title}{}{}{}

\renewcommand{\abstractnamefont}{\sffamily}
\renewcommand{\abstracttextfont}{}
\renewcommand{\absnamepos}{flushleft}


\makechapterstyle{mestrado}{% Originally ``AlexanderGrebenkov'', adapted
\renewcommand{\chapterheadstart}{\goodbreak\vspace*{\beforechapskip}}
\renewcommand{\chapnamefont}{\normalfont\Large\scshape}
\renewcommand{\chapnumfont}{\normalfont\Large\scshape}
\renewcommand{\chaptitlefont}{\normalfont\Large\scshape}
\renewcommand{\printchaptername}{}
\renewcommand{\chapternamenum}{}
\renewcommand{\printchapternum}{\normalfont\Large\scshape\S\thechapter}
\renewcommand{\afterchapternum}{\hspace{1em}}
\renewcommand{\afterchaptertitle}{\par\nobreak\vspace{-.9em}\moveright 6pt\vbox to 1pt{\hrule width .32\textwidth}\nobreak\vskip\afterchapskip\nobreak}
}
\chapterstyle{mestrado}

\begin{document}

\maketitle

\begin{abstract}
In July 2010, we worked with a group in Mozambique, Kufunana, which uses
theatre, particularly theatre of the oppressed, to reach out on HIV/AIDS
issues. We directed a show with their actors, a show we developed with them,
loosely based on Arthur Schnitzler's play ``La Ronde.'' We also took the
opportunity to talk to them and interview them about their work more generally.

In Mozambique, a very poor country, theatre is regularly used by many groups to
reach out with their social messages.

\end{abstract}

\chapter{Introduction}

\chapter{A Rede (The Network)}
We described the basic concepts of Schnitzler's La Ronde to the elements of
Kufunana. We presented it not as a goal (to perform an European play) but as a
set of initial concepts for discussion and asked for feedback. Some of the
elements of Kufunana related it to a recent concerted effort by HIV-related
NGOs to make people aware of the sexual networks of which they are a part of.
Thus the idea developed from the circle to the network.

All of our scenes emerged organically through improvisation with the actors.
After our initial discussion setting the general guidelines, we had actors
improvise over these themes. Actors were not assigned roles at this stage and
some themes (for example, teacher and female student) were played by several
actors. The actors had a mandate to explore Mozambican themes and characters,
but, beyond that, were free to improvise.

In addition to on-stage work, we had discussions with the actors on whether the
piece represented Mozambican society and what was missing. For example, at one
point, we realized that we had a rich man, but no politically powerful figures
and this led us to add a member of parliament.

From La Ronde, we kept the structure that almost all of the scenes to be
partner scenes with a sexual encounter towards the end. We had characters from
all strata of society.  While it does not make sense to have a count in modern
Mozambique, we had a member of parliament, and a rich business to represent the
ruling classes. The only character who is present in both versions is, perhaps
not surprisingly, the prostitute.

We had an imbalance in the distribution, with more male actors than female
actors. Partially for this reason, but also to introduce outside observers that
could comment on the actions of the other characters, we introduced what we
called a ``Greek choir of drunkards.'' Their goals and actions were envisioned
to be similar to the Greek choir (to foreshadow events, to tell the back story,
and to provide the \textit{vox popolis}), but to introduce a Greek choir would
be a completely foreign element to Mozambican culture and would not be
understood by the audience. Therefore, we had the men of the choir sit at a
bar, drinking and talking about the actions on stage. Their inebriation served
as the justification for their frankness about everyone in the community.

\chapter{Kufunana's Work}

Directing this show was our most direct experience with Kufunana, but their
work is more broad. They often use theatre of the oppressed techniques in work
places around the city to attempt to get workers tested (or, if necessary,
treated) for HIV and provide counseling. This is actually one of their main
sources of income as companies will often pay for their services as they have
now recognized that keeping their employees in good health is in their best
interests.

Additionally, they often traveled to the country-side to do their work. Their
typical approach is to combine theatrical performances, theatre of the
oppressed, pure entertainment, and discussions.

\chapter{Conclusions}
We have described our work with Kufunana.

In a collaborative effort, we adapted a century old play about \textit{fin de
siecle} Vienna to modern day Mozambique. As outsiders, we brought this idea to
the group, but let their responses organically mold the show as we would be
otherwise unable to create something that spoke to the community. As a side
effect of creating this show, we learned much about the structure of Mozambican
society, particularly its sexual economy, that the actors would portrait on
stage and which we would then follow up with discussions.

Theatre in Mozambique is used routinely to serve social goals.

\end{document}
