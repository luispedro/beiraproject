\documentclass[article,twocolumn,twoside]{memoir}
\usepackage[utf8]{inputenc}
\usepackage{psfrag}
\usepackage{hyperref}
\usepackage{color}
\usepackage[]{graphicx}

\graphicspath{{../figures/other/}{../figures/generated/}}


\title{Theatre for Development: Theatre Classes in Rural Mozambique}
\author{Rita Reis and Luis Pedro Coelho}

\bibliographystyle{alpha}
%\bibliographystyle{plain}
\pagestyle{Ruled}

\aliaspagestyle{chapter}{Ruled}
\makeatletter
\if@twoside
	\makeoddhead{Ruled}{}{}{\scshape\rightmark}
	\makeevenhead{Ruled}{\scshape\leftmark}{}{}
	\makeevenfoot{Ruled}{\pagenumberfont\thepage}{}{}
	\makeoddfoot{Ruled}{}{}{\pagenumberfont\thepage}
	% Put section number on top
	\def\sectionmark#1{\markright{#1 (\thesection)}}
	\def\chaptermark#1{\markboth{#1}{#1}}
	\renewcommand*{\bibmark}{\markboth{\bibname}{\bibname}} % I don't like empty headings!
\else

	% Put section number on top
	\def\sectionmark#1{\markright{#1 (\thesection)}}
	\def\chaptermark#1{\markright{#1 (\thechapter)}}

	\makeoddhead{Ruled}{}{}{\scshape\rightmark}
	\makeevenhead{Ruled}{}{}{\scshape\rightmark}
	\makeevenfoot{Ruled}{}{}{\pagenumberfont\thepage}
	\makeoddfoot{Ruled}{}{}{\pagenumberfont\thepage}
	\renewcommand*{\bibmark}{\markright{\bibname}} % I don't like empty headings!


\fi
\makeatother

%Change fonts: Page number sans-serif (much cleaner than the roman version)
\def\pagenumberfont{\sffamily}
% Change fonts: Section headers Sans-Serif:
\setsecheadstyle{\Large\sffamily\raggedright}
\setsubsecheadstyle{\large\sffamily\raggedright}
\setsubsubsecheadstyle{\normalsize\sffamily\raggedright}

% Title
\pretitle{\LARGE\sffamily}
\posttitle{\par\vspace{4ex}}
\preauthor{\large\sffamily\hspace{1cm}}
\postauthor{\par\vspace{3ex}}
\predate{\small\sffamily\hspace{1cm}Last Updated on: }
\postdate{\par\vspace{2cm}}
\copypagestyle{title}{plain}
\makeoddfoot{title}{}{}{}
\makeevenfoot{title}{}{}{}

\renewcommand{\abstractnamefont}{\sffamily}
\renewcommand{\abstracttextfont}{}
\renewcommand{\absnamepos}{flushleft}


\makechapterstyle{mestrado}{% Originally ``AlexanderGrebenkov'', adapted
\renewcommand{\chapterheadstart}{\goodbreak\vspace*{\beforechapskip}}
\renewcommand{\chapnamefont}{\normalfont\Large\scshape}
\renewcommand{\chapnumfont}{\normalfont\Large\scshape}
\renewcommand{\chaptitlefont}{\normalfont\Large\scshape}
\renewcommand{\printchaptername}{}
\renewcommand{\chapternamenum}{}
\renewcommand{\printchapternum}{\normalfont\Large\scshape\S\thechapter}
\renewcommand{\afterchapternum}{\hspace{1em}}
\renewcommand{\afterchaptertitle}{\par\nobreak\vspace{-.9em}\moveright 6pt\vbox to 1pt{\hrule width .32\textwidth}\nobreak\vskip\afterchapskip\nobreak}
}
\chapterstyle{mestrado}

\begin{document}

\maketitle

\begin{abstract}

Theatre games - observing Mozabican society

In June and July 2010, Rita Reis and Luis Pedro Coelho, traveled to Mozambique
and taught theatre classes to high school students in four rural schools. Each of the
seven workshops was a week-long affair, with students working 1h30 per day.

Most of the students involved were already participating in school theatre
groups, which in the Mozambican context is closely linked to working as
volunteers for different organisations to spread information on topics such as
HIV/AIDS, other STDs, or education.

During this workshops, and especialy through the improvs and discussions after the
improvisational exercises, Rita and Luis where able observe some of the
specificities of the Mozambican society. For the students those exercises where
an oportunity to develop their skills as actors and a space to question some
habitual behaviour paterns.

\end{abstract}

\chapter{Introduction}
In June and July 2010 we taught theatre classes to high school students in four
rural schools in the provicie o Sofala in Mozambique. These were run by
EsMaBaMa, a non-profit Catholic organisation and are located in Estaquinha,
Mangunde, Baranda, and Machanga (hence the name, EsMaBaMa), all in the Sofala
provincy of Mozambique. Sofala is one of the porest regions of the world, and
it has a rate of HIV-AIDS infection of thirty percent.

All of the schools already had some theatre structures. Sometimes these were
run by the Peace Corps volunteers and performed in English in order to get the
students to practice that language. But most ot the theatre groups worked in
Portuguese.

Our course had several goals: to raise awareness on HIV-AIDs, to support the existing
theatre structures by teaching basic exercises the groups could continue to use
after we left, to help support Portuguese class by reiterating to the students
some of the grammatical and rhetorical concepts, in some cases, to have the
students practice their English, finally, to make it fun for the students. The
programme that we followed evolved throughout our stay as we observed and
reflected on the results of the previous experiments. And we discovered by
doing so that reharsed imprrovisations were a particularly interesting exercise
that revealed habitual paterns of behavior charcteristic of the Mozambican
society.

During the workshops we did warm-up exercises, reharsed improvisations, text
work with poetry by two Mozambican well known authors Jose Craveirinha and
Mia Couto and an introduction to Commedia Dell'Arte's charaters and
improvisations.

\section{Commedia Dell'Arte}

Theatre in Mozambique is often based on improvisations, reharsed improvisations
without a writen text, actors are very physical and portrait big character
types - such as the spoiled little girl or the older greedy man. There are so
many similarities with Commedia Dell'Arte that we decided to spend a little
time doing introduction to Commedia Dell'Arte. The goal was to help the
students to broden and physically develop their palet of characters. And the
traditional \textit{comedia dell'arte} form, which we will refer to as simply
\textit{commedia}, gave us many characters that are physically well-defined and
expansive.

We took the time to introduce and play with the shapes and wants of the
characters. And when the students had the Commedia characters in their bodies
we started to work on structured improvisations. In particular, the
\textit{Capitano} meets the \textit{Signora}, she attempting to get him to
propose her\footnote{or to convince him to have unprotected sex with her}, he
attempting to get a meal while pretending to be rich.\footnote{The
\textit{capitano} In this improv he lost his fortune, but refuses to admit it.
Therefore, he must be boastful while living off others. The \textit{signora} is
an older lady with some money who is desperate for marriage.}

This improvisation was always very successful, the students seemed to
particularly connect with the wants and shapes of those two characters.
Capitano wanted food and money, Signora love and sex, they where both extremely
vain, and their shapes with big and sexual. All those characteristics were a
particularly interesting cocktail that captivated the young Mozambican
students. This structured improvisation enables lazzi that are recognisable in
italian and western cultures, but has we discovered during this work extremely
efective with teenagers discovering sex and seduction and living in a country
where hunger is a regular threat.

\section{Improvisations}

A big part of the workshop where the reharsed improvisatios. Those where based
on the method the students use to create pieces in their theatre groups. But we
decided to make improvistions on specific locations that are part the students
everyday life to see what topics would come up. Group improvisation settings
were ``chapa'', health care centre, market, and party.

The Chapa was one of the most dynamic improvisations. In many African
countries, the primary means of public transportation are 12-person minivans
which regularly carry up to 20 passengers crammed into its seats. In
Mozambique those minivans are called Chapa, they are used both in the
city\footnote{In the city, the chapas are actually a very convenient way to get
around as the frequency is very high.} and inter-city. The ticket collector,
who also calls out for passengers whenever the chapa stops (chapas stop on any
intersection as long as someone inside or outside calls out, but there are
semi-established stops), served as the leader in this improvisation. The
sceene started with the ticket collector alone on stage and who would call out
to the rest of the participants to join him. When he had assembled a sufficient
number of passengers, he'd call out for a driver (this mimics the behaviour of
actual chapas at terminal points when the driver might step out for a little
break as the ticket-collector gathers new passengers). Other characters include
salespersons, traffic police (who, this being Mozambique, were willing for a
bribe), the beggar, the chapa owner, and, in one inspired scene, the man ran over
by the chapa. We describe it here at length as it was invariably successful in
terms of both including everyone, being fun, and generating interesting scenes.
Therefore, we often used it as the first improvisation of the workshop.

The party improv was mostly interesting for us as it revealed many of the
typical teenage flirtation modes of Mozambican teens. There would be the DJ,
the barman and all the guests. Those where inspired in a parties that regularly
took place in the schools on sundays after church. During our stay we have had
the oportunity to atended one of those.

The health care centre featured doctors, patients, receptionists, pharmacists,
and a helper. The schools where we did the workshop were often located very
close to a health center and it was a very familiar place for the students
since most of them lived in dorms away from their parents and where often sick.
The receptionst was be the central charater was the ticket collector had been
in the chapa the improvisation would start in the beguining of the day the
receptionist alone and then very rapidly a long line of patients would start to
take shape. Patients would go through several stages of burocracy and in the
end finaly get their medication for diarea, malaria or hiv-aids

The market is a good improvisation for continued action. The informal markets
are were most people go to buy anyting they might need, the number of actual
shops is verry limited even in the city. There would be vendors, costumers,
begars, thieves and prostitutes. And always trouble for finding change, since
the vendors are afraid of carrying cash.

\section{Mozambican Societal Issues unveild by the iprovisations}

Mozambican society had many problems which showed up again and again the
improvisation work. To some extent, this is the result of the role that theatre
plays in Mozambican society, very often focused on social issues; but it is
also a reflection of a society with many grave problems. The students sometimes
acted them out to put the spot light on something the thought was worth being
questioned and criticized during the improvisation. Othertimes though some of
behaviour patterns where so habitual, that even when the the intructurs pointed
out how opressive those behaviours could be, the students would have trouble
undrestanding it

The corrupt traffic policeman was a staple of the chapa improv. In some cases,
he would ask for money directly; other times, the driver would offer it. We
sometimes suggested that a policeman would show up, but we never suggested he
would be corrupt---that the students introduced themselves. In fact, a traffic
policeman asking for documents is generally assumed to be looking for
bribes.\footnote{In one occasion, while in the city, the chapa we were riding
was pulled over by a policeman asking for documents. We watched, with interest,
having seen this scene played out by the students in improv. Sure enough, the
policeman kept the driver's license, the ticket-collector got out a few yards
further, and ran back to get the documents (certainly, not for free). Even if
anecdotally, we can report that traffic policeman do look for bribes in
Mozambique.}

The women's position and relationship to sexuality and money was also
repeatedly portrayed as one of business. This is an issue that is extremely
present in Mozambican society, with many relationships between men and women
straddling the line between gift offering and outright prostitution.

In "The Party" improvisation seduction was often around the direct statement of
``I'm rich, I can buy you presents'' (or conversely, the female refusal was of
the form ``I'm not interested. I already have a boyfriend and he's
rich.'').\footnote{Another interesting exchange took place in a 3-person improv
we directed in another context. The male element in a couple tries to bribe a
policeman with a small bill, saying ``it's all the money I have;'' to which the
policeman replies by turning to the girl and asking ``how can you date a guy
who has no money? He just tried to give me a ridiculous bribe!''} This pattern
showed up even when we asked the students to play switched genders. In another
case, a woman was prostituted in the market with one of the male students
playing an intermediary who promised to take a small cut, but then took all of
the money for himself.

These are all issues that are present in Mozambican society, with many
relationships between men and women straddling the line between gift offering
and outright prostitution

HIV/AIDS was present too. Not only where it was obvious (health care centre),
but everywhere a condom salesman would show up. In ``market''
improv one even included a thorough explanation of how to use a condom.\footnote{In
this case, we actually appropriated this scene and re-fashioned it for the
``chapa'' improv, where the salesman would approach the bus as it stopped. This
was part of the presentation shown to the school.} The students were all
very knowledgeable of the issues and, in a way that in the West would be very
surprising for their age, able to discuss it in front of others in a completely
matter-of-fact way. One improv, which we set up as ``sexual education class''
was reminiscent of that Monty Python sketch where the teacher, played by John
Cleese, explains sex to a bored teenage audience who is as thoroughly
uninterested as they would be in lessons in Ancient Greek. In our case, the
students simply answered the teacher's questions as good students would answer
questions on History or some other subject. There was little tension in this
improv and we did not use it again.

In one ocasion ``The Healh Clinic'' improvisation has generated a conversation
about the fact that in the health care center ran by the catholic there where
free condoms availiable for the patients in general but not for the religious
school students. But the students said that even though the rules where strict
concerning that matter they found ways to hide their condoms and get them
elswhere. And a couple of students even showed the condom they where carying
with them. This situation was paradigmatic of what we heard about catholic
church missions in mozambique and their position thowards the usage of condoms.
They are for it, they distribute condoms but they don't want to make a big fuss
out of it to avoid conflicts with hierarchies.

Sometimes the students reacted against an overly negative portrayal of
Mozambique with the argument that they shouldn't only show the problems in
society but also that there are good things too. This led us to explore
alternative choices with the characters. In Estaquinha, we played a repeated
scene in which three friends meet after taking an HIV-test. One of them is
positive and is alternatively rejected by his ignorant friends (``that's why
you have those pimples! Even those ripped trousers are part of the disease! I
can't talk to you! We'll get contaminated.'') or embraced and correctly
counselled by them (``Don't look that discouraged! You'll outlive us all! Just
take your medication. And we'll still be your friends, come, let's have a
drink!''). These two scenes were worked into two improvs that were presented to
an audience one after the other.

Presntation - Adudience Reaction
On one occasion, the students presented to the school community, a very tough
audience, who would laugh at a student's mishaps. The students varied in their
preparedness, given that some had not even attended all of the sessions. We
gave everyone an opportunity if they felt that they were ready, but we tried to
give them a longer or shorter time on stage depending to avoid, as much as
possible, that they would feel embarrassed. We had previously worried that our
intuitions of what is a good performance would not match what a Mozambican
audience would enjoy, but we were heartened to find that the audience agreed
with us as to which of the performances were best. We where surprised to see
that there was no culture gap un the humour and quality of the pieces between
the Mozambican audience's opinion and ours.

\chapter{Challenges}
\section{Language}

Mozambique is home to many languages. In the region of Sofala where we were
working, Ndau is the most common language (Sena is also spoken in Sofala).
Portuguese is the official language of the country, but, particularly in the
country-side, some students only encounter it when they reach school age. In
the city, Portuguese is current, and sometimes a first language. Given that
some of our students came from the country side, others from the city, and from
different background, their language skills varied widely and some students
struggled with vocabulary.

\section{Women's Position in Mozambican Society}

Mozambican society is highly patriarchal. This shows up both in the content
of the work (see below), but also in the way that women worked. In some groups,
there were very charismatic, talented female students, of course; but, on
average, the female students tended to be less self-confident and come forward
less. We also noticed that women knew that even though they where in the same
classes women laked a lot of language and grammar concepts compared to the men.
Many where not able to identify the verb in a sentence. This tremendous
difference, in the education level of the male and female participants of the
workshop, is probably related to the fact that in Mozambique women are expected
to marry very young and to take care of the house and children, not to work.
And they are probably not as motivated as men to get an education and not
pushed as hard in school.

In one occasion, a female student volunteered to play ticket-collector in the
``chapa'' improv, to which some of the male students reacted negatively saying
that a ``woman cannot be ticket-collector.'' This discouraged the girl and even
after we said that she could play the role, she was no longer interested.

In general, female students, perhaps internalising these relationships were
less forthcoming than male students when we asked for volunteers or for someone
who was ready to present their work. That is why we started to insist on
having female volunteers, and through positive reinforcement we to diligently
encourage the girls who did volunteer.

\section{Structured Work}

Given the voluntary nature of the work and Mozambican attitudes, students were
often late. We attempted to always book a two hour slot, but keep the session
to exactly ninety minutes from the moment we actually started. We informed the
students of this fact. It was very helpful if one of their teachers was
present, but this was not always possible. We decided to start when we felt we
had enough of a critical mass to avoid waiting too long or not being able to
end on time, but this meant that there were students trickling in for a long
time.

Similarly, it often happened that a student would wish to join the group
half-way or even at the last session. We enjoyed the appreciation that this
represented (as opposed to the more typical waning of enthusiasm and
attendance, in some cases, total attendance went up) and we tried to always
accommodate them, but it sometimes slowed progress.

\chapter{Assets of the Mozambican high school theatre groups}
As well as a unique set of challenges, Mozambican students had many assets when
compared to Western students of the same age. And it was very exiting to
discover them.

Theatre and poetry are important parts of Mozambican culture. We were fortunate
to be present for the celebration of Independence Day (June 25). The
celebration in Machanga (a very small town) consisted of speeches, live and
recorded music, and several egaged poetry readings.

Many of our students were \textit{activistas} in various youth groups. While in
European Portuguese the word \textit{activista} has the same connotation as the
English word \textit{activist}, in Mozambique, the expression has wider usage
to denote organised volunteering. In fact, there is widespread youth social
participation in organised groups such as \textit{JOMA}, Youth for Change and
Action,\footnote{\textit{Jovens Para Mudança e Acção}, in Portuguese.} which
organises around HIV related issues and was present in every school we visited.
Participation in these groups is taken seriously and is a source of pride for
the \textit{activistas}. This seemed to us to be a feature of Mozambican
society and not only for younger students---older people also took pride in
participating in civil society activities.

There is also a national competition on English language theatre for high
school students, which is further motivation that students have for
participating in theatre groups. Therefore, our work was shaped to serve these
existing structures in their activities rather than to build up new structures.
The teachers who organise these groups were also invariably enthusiastic and an
indispensable help in handling logistic details and interfacing with the
students and school.

At each location, there was always an enthusiastic core of students who
participated and were committed. They took responsibility for the rest of the
group and were often naturally in the role of serving as an interface between
us, the other students, and the school---which was invaluable when their
teacher could not be present and we did not know whom else to approach for some
practical issue.

On stage, the students were freer with their bodily expression and less
self-conscious than we feel Western students of the same age and theatrical
experience are. They had less social inhibitions to moving freely with their
whole bodies. We also remarked that the male students were not afraid to play
female or effeminate characters.

As a counterpart to the lack of rigour in preparation, the students are
very capable of improvising both in class and in front of the
audience.\footnote{In Estaquinha, when we presented in front of an audience, we
were both laughing at the new jokes that the students came up with on the
spot.} They also did not feel stumped by a lack of rigorous preparation before
presenting (which, due to time constraints, we did not have).

The theatre groups run by the students generally work with ``prepared
improvs.'' A scene will start out as an improvisation based on some general
theme or set of characters, they will pick the best lines, rehearse it a few
times (but not more), until the general lines of action and dialogue become
clear, and then present. Still, even when presenting, the text is not
completely set, and is never written down anywhere. Therefore, they have
experience and an intuitive understanding of this type of work and it was
generally very good.

\chapter{Conclusions}

We taught this class to over 150~students in four different rural Mozambican
schools. The class evolved as we experimented different ideas and exercises.

We were fortunate that we were able to teach this in the context of existing
groups of \textit{activistas} involved in theatre and, therefore, some of the
work served as support to an existing structure.

Knowing the common language of the country was essential. Even the English
theatre groups were run in Portuguese with only the performances itself being
in English. When describing an improv which was to be in English, we used
English to get the group to start thinking in English, but were sometimes
forced to translate at least some of the words.

The programme changed as we gained experience and we have, so far, mostly
described our final ideas. Some exercises were found to work better than others
and our thinking changed. We narrowed down on a smaller set of exercises than
we started with under the reasoning that it was best to repeat the same
exercise several times so that the existing theatre groups could keep a memory
of that exercise and use it in the future.

We had read a lot about Mozambique, were quite prepared for what we encounterd
on site, and had chosen a large number of exercises we were hoping o use. But
there where some of aspects of the culture, that we only discovered in the
classroom. That is why, for example, we chose improvisations using locations
that where specific to the Mozambican reality, like "The Chapa". That is also
why after seing the boold and phisical improvisations of the students we choose
use commedia de l'arte characters and improvisations.

We have learned a lot from this exchange. In western countries we often hear
that theatre is a dead art but for the Mozambicans we worked with theatre is a
vital vehicule to educate about the issues their society faces concerning
health in general, HIV-AIDS, cholera, education and gender roles. 

We think our experience can be informative for people who are curious about
Mozambican theatre and society. Such a different reality and way to do theatre
is surprising and interesting for those who ere curious about how our our form
of art used as an educational tool in other parts of the world. It can also be
interesting for to people who want to do theatre in developing countries, or
with a recent immigrant populations that are originally from developing coutries.
Our work might be interesting for those people because the challenges and assets
may slightly vary but poverty, gender role issues, corruption, aids, and some
of the other societal issues we described in our paper will still invariably
show up in the work. Hopefully our paper will be an example of the fact that
rehearsed improvisations in the format we used in Mozambique can be an
interesting tool to generate reflection on Societal Issues.

\end{document}
